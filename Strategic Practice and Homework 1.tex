%This is my super simple Real Analysis Homework template

\documentclass{article}

\usepackage{xeCJK}  % 中文
\usepackage{enumitem} % 标签 (a) (b) (c)
\usepackage[dvipsnames]{xcolor} % 颜色
\usepackage{ulem}  % 划掉文字
\usepackage{cancel}  % 花掉公式
\usepackage{indentfirst} % 缩进
\usepackage{amsmath}  % 数学符号
\usepackage{amssymb}



\title{Strategic Practice and Homework 1}
\author{Wen}
\date\today


\begin{document}
\maketitle %This command prints the title based on information entered above

%Section and subsection automatically number unless you put the asterisk next to them.
\section*{Strategic Practice}
\subsection*{Problem 1}
\begin{enumerate}[label=(\alph*)]
\item $>$

4个骰子最多有24点,题目中21点与22点与这个数字比较接近,所以应该是用数数的方法。

对于21点,有 6 6 6 3,6 6 5 4,6 5 5 5,将四个骰子进行编号,那么 6 6 6 3 可能情况有4种,6 6 5 4 可能情况有12种,6 5 5 5 可能情况有4种,一共20种。

对于22点,有 6 6 6 4,6 6 5 5,将四个骰子进行编号,那么6 6 6 4可能情况有4种,6 6 5 5可能情况有6种,一共10种。
\item \sout{$<$} \textcolor{red}{=}

\sout{两个回文字母,要求第一个字母与第二个字母相同,则一共有26种可能。

如果是三个字母是回文,则要求前2个字母相同,中间字母任意,则一共有26 * 26 种可能}

\textcolor{red}{上面的错解来源于数个数时,但是2个事件每个的概率是不一样的,所以不能单独数个数,换一种想法,两个事件都是要求前后两个字母相同,则概率是相同的。也可以用indicator来想这个问题}

\end{enumerate}

\clearpage %Gives us a page break before the next section. Optional.


\subsection*{Problem 2}
\begin{enumerate}[label=(\alph*)]
\item flush表示5长牌颜色相同,花色有四种,但是不含不能是10 J Q K A

$$ \frac{4\left( \left( \frac{13}{5} \right)  -1\right)  }{\left( \frac{52}{5} \right)  } $$

\item \sout{有两对,也就是13张牌中选3张牌,然后再从中间选出来2张牌作为2对}


$$
\cancel{
\frac{\left( \frac{13}{3} \right)  \left( \frac{3}{2} \right)  \  }{\left( \frac{52}{5} \right)  } 
}
$$


\textcolor{red}{上面的例子错再没有考虑到每种牌有花色的问题,2张牌有6种颜色,另外2张也有6种颜色,单独的一张有4种颜色,一共144种结果,错解乘以144就是正确答案}

$$
\textcolor{red} {
\frac{\left( \frac{13}{3} \right)  \left( \frac{3}{2} \right)  \left( \frac{4}{2} \right)^{2}  \left( \frac{4}{1} \right)  }{\left( \frac{52}{5} \right)  } 
}
$$

\end{enumerate}
\clearpage



\subsection*{Problem 3}

\begin{enumerate}[label=(\alph*)]
\item 向右走了111步,向上走了110步,总共221步,也就是在221步种选111步向右

$$ 
\left( \frac{221}{111} \right)  
$$

\item 这道题目可以两步走,第一步到达(110,111),第二步到达(210,211)。用乘法规则来处理。

$$
\left( \frac{221}{111} \right)  \left( \frac{200}{100} \right)  
$$
\end{enumerate}

\clearpage

\subsection*{Problem 4}
假定如果只有一个字母,有26种可能,如果有两个字母有26 * 25种可能,以此类推,一共有26 + 26 * 25 + 26 * 25 * 24 + ...... + 26!, 每种情况的概率都相同。所以使用所以的26个字母的概率就是

$$
\frac{26!}{26\  +\  26\  \times 25\  +\  ...\  +\  26!} \  =\  \frac{1}{1\  +\  1!+\  \frac{1}{2!} \  +\  \frac{1}{3!} \  +\  ...\  +\  \frac{1}{25!} \  } \  
$$
我们发现分母的结果其实与e的泰勒展开接近


\clearpage
\subsection*{Problem 5}

\begin{itemize}
\item 我们要做的事情是从n个物品中随意挑选物品
\item 左边代表先规定要取多少个物品,然后再取物品
\item 右边代表有n个物品,每个物品可以选或者不选。
\end{itemize}

\clearpage

\subsection*{Problem 6}
左边发现是2n的全排列,除以n!,表示有n个东西的全排列重复了,再除以$2^{n}$,表示这n个东西内部还有排列重复了,总结来看,就是将2n个东西,分成配对的n组。带结果值尝试一下,当n=2时,左边等于3。这与把4个物品,分成配对的2组,结果一样。
\begin{itemize}


\item 我们要做的事情是将2n个东西,分成配对的n组。
\item 对于左边,每一种配对的情况,对应全排列重复了$2^{n}\times n!$次
\item 右边,这里不会
\end{itemize}

\textcolor{red}{右边可以想对所以人进行编号,1号可能和n-1个人进行配对。再找2号(也可能是3号),可能和n-3个人进行配对,用树的思想更好理解这个问题}

\clearpage
\subsection*{Problem 7}

\begin{itemize}
\item 我们要做的是,在n+1个人中,选择k个人
\item 对于左边,我们假定对所有人进行编号,最后取出的k个人中,分为含有1号和不含有1号两者情况,当含有1号时,有$\left( \frac{n}{k-1} \right) $种可能,当不含有1号时,有$\left( \frac{n}{k} \right) $种可能性。
\item 右边就是上面所说的
\end{itemize}

\clearpage


\section*{Homework 1}
\subsection*{Problem 1}
这个事件等价于6个人中选3个人,3个人选到的都是女生。
$$
\frac{1}{\left( \frac{6}{3} \right)  } 
$$

\clearpage
\subsection*{Problem 2}
\begin{enumerate}[label=(\alph*)]
\item 这是一个典型的分堆问题,在国防科技大学概率论课堂有详细的说明,是用递推的思维来想的,先分一堆,再分一堆,但也要注意重复数的问题,例如分成相同的两堆,最后是要除以2的。

哈佛大学给出的答案,就是把全排列去除以重复的次数,\textcolor{red}{反过来想,就是一次分堆,对应多少全排列}

$$
\left( \frac{12!}{2!5!5!\cdot 2} \right)  
$$

\item 与上题类似,\textcolor{red}{反过来想,一次分堆,对应多少次全排列}
$$
\left( \frac{12!}{4!4!4!3!} \right)  
$$

\end{enumerate}

\clearpage


\subsection*{Problem 3}
这个问题相当于有10个框子,有3个球,向框子里面仍球。给框子编号,给球编号。一共可能有$10^{3}$种可能。若要球不进入同一个框子,有10*9*8种可能。所以有重叠的概率是 
$$
1\  -\  \frac{10\  \times \  9\  \times \  8}{{}10^{3}} 
$$

\clearpage

\subsection*{Problem 4}
这个问题相当于有6个框子,有6个球,向框子里面仍球。当每个框子都是1个球的时候,才能保证每个框子中的球小于2

$$
1-\frac{6!}{6^{6}} 
$$

\clearpage

\subsection*{Problem 5}
第一次捕获后,麋鹿被分为了被标记和没有被标记。相当于从n里选出k个,从N-n里面选出m-k个。但也要考虑范围。

$$
\frac{\left( \frac{n}{k} \right)  \left( \frac{N-n}{m-k} \right) } {\left( \frac{N}{n} \right)  } 
$$
这个就是超几何分布
\clearpage

\subsection*{Problem 6}
\begin{enumerate}[label=(\alph*)]
\item 由于第一个人拿出球后没有看,先后拿与拿出来2个球,分给两个人的概率是相同的,所以两个人的概率是相同的

\item 对所以的球进行编号,第一次拿出来是绿球的概率,很简单,是
$$
\frac{g}{r+g} 
$$
第二次取球,可以用树来做,一共有$\left( r+g\right)  \left( r+g-1\right)  $种情况。第一次取到红球,第二次取绿球,一共 $r\cdot g$种情况。第一次取到绿球,第二次取到绿球,一共$g\left( r+g\right)$种情况。最后结果为
$$
\frac{r\cdot g+g\left( g-1\right)  }{\left( r+g\right)  \left( r+g-1\right)  } =\frac{g}{r+g} 
$$
可以看出,两个结果是相同的

\item 这里可以用条件概率,设事件A为第一次红色,B为第二次红色。
$$
P\left( AB\right)  +P(A^{C}B^{C})=P\left( A\right)  P\left( B|A\right)  +P\left( A^{C}\right)  P(B^{C}|A^{C})\  =\  \frac{r}{16} \cdot \frac{r-1}{15} +\frac{16-r}{16} \cdot \frac{15-r}{15} 
$$

$$
P(AB^{C})+P(A^{C})P(B)=2\cdot \frac{r}{16} \cdot \frac{16-r}{15} 
$$
令上面上个式子相等,解出$r=6$或$r=10$

\textcolor{red}{还有更快的方法:由于所求两个事件互为对立事件,并且概率相等,所以上面的式子值为$\frac{1}{2}$}
\end{enumerate}

\clearpage
\subsection*{Problem 7}
\begin{enumerate}[label=(\alph*)]
\item 
这道题其实用递归的思想可以解,类似于前面的story proofs第7题
$$
\left( \begin{gathered}n+1\\ k+1\end{gathered} \right)  =\left( \begin{gathered}n\\ k\end{gathered} \right)  +\left( \begin{gathered}n\\ k+1\end{gathered} \right)  =\left( \begin{gathered}n\\ k\end{gathered} \right)  +\left( \begin{gathered}n-1\\ k\end{gathered} \right)  +\left( \begin{gathered}n-1\\ k+1\end{gathered} \right)  =...=\left( \begin{gathered}n\\ k\end{gathered} \right)  +\left( \begin{gathered}n-1\\ k\end{gathered} \right)  +...+\left( \begin{gathered}k\\ k\end{gathered} \right)  
$$
我们的故事就可以这样编,选出一个最年长的人,将所有含有他的结果先计算出来,为$\left( \begin{gathered}n\\ k\end{gathered} \right) $,在剩下的n个人中,在选一个最年长的,将含有他的结果都计算出来,为$\left( \begin{gathered}n-1\\ k\end{gathered} \right)  $,以此类推,最后剩下的k+2个人的时候,选一个最年长的,将含有他的结果都计算出来,为$\left( \begin{gathered}k+1\\ k\end{gathered} \right)  $, 然后最后只剩下k+1个人,就是最后一种情况。

\item 一共假设有N gummi bears,中间加4个隔板就可以将5种口味隔开,那么总共有N+4个点,在其中选择4个点作为隔板,就能得出有多少种口味的组合。
$$
\left( \begin{gathered}N+4\\ 4\end{gathered} \right)  
$$

\textcolor{red}{题目说的是N从30到50的一个求和,总共多少}


$$
\textcolor{red} {
\sum^{50}_{i=30} \left( \begin{gathered}i+4\\ 4\end{gathered} \right)  =\sum^{54}_{j=34} \left( \begin{gathered}j\\ 4\end{gathered} \right)  =\sum^{54}_{4} \left( \begin{gathered}j\\ 4\end{gathered} \right)  -\sum^{33}_{4} \left( \begin{gathered}j\\ 4\end{gathered} \right)  =\left( \begin{gathered}55\\ 5\end{gathered} \right)  -\left( \begin{gathered}34\\ 5\end{gathered} \right)  
}
$$

\end{enumerate}
\end{document}